% preface.tex:

\begin{preface}

Here I'll write a dope section. More personal. Something that states why I started this research.

% BRING IT HOME!
%This dissertation is an opportunity to course correct. Nearly 100 years passed between the first demonstration of x-rays and the first whole-body MRI image. In the 45 years since, we are still using potentially harmful radiological and mediconuclear approaches and limiting our understanding of ourselves through modalities that are not conducive to the unrestricted measurements needed globally today. As a medical community owe it to ourselves to reflect on the use of our innovations to ensure a viable healthy future. With this in mind, the call to action of this thesis is to turn our attention to new   Enter: optical imaging. 
%xyra - 1895
%ultrasound - 1956
%MRI - 1977 (like human patient)
%PET - 1974
%82 years between these inventions. Thats how long it took for us to step away from harmful, ionizing imaging to MRI. 
%45 years to date. Accelrating innovation. 


%Fascination with seeing inside people. C-rays are the things of comic books. This is real. And light doesn’t care about the trivial things of society- it works for everyone and all languages. That’s the reason for the grand challenges chosen, and the reason why the abstract, the grand summary of my work, is also written in Spanish. An homage to my lineage and those in my family who helped in one way or another, make this work successful. They should, like all, have personalized access to this work.

%Which is why I appreciate art. Language less interpretation. It’s the human language. I highly recommend you experience light. Whether a sunrise or sunset, or a museum.

\end{preface}

