% intro.tex:

\chapter{INTRODUCTION} % all caps please
\label{chap:introduction}

% WHY MEDICAL IMAGING?
Modern civilization has leveraged medical imaging as a fundamental clinical and research tool for years. Although x-rays dominated the field for 80 years after its invention~\cite{Gunderman2012}, the field is growing at a rapid pace due in part to the increasing availability of relatively inexpensive computational resources~\cite{Iglehart2006}. In recent decades, we have seen an emergence of new imaging technologies that improve on traditional methods being developed and commercialized, including \ac{MRI}, nuclear imaging [such as \ac{PET} and \ac{SPECT}], and ultrasound imaging~\cite{Suetens2017}. These contemporary imaging modalities have improved on the ionizing approach of x-ray, and thus, more and more frequently have taken the center stage of routine clinical use. 

Unfortunately, many of our modern-day diagnostic approaches rely heavily on radiation- and nuclear-based imaging tools---so much so that the largest man-made source of radiation exposure comes from radiation due to medical examinations~\cite{Picano2004}. Nuclear imaging relies on the detection of injected radioactive isotopes that attach to biochemically active substances in the body~\cite{Fahey2002}. Improving on these methods, ultrasound imaging uses high-frequency sound waves that are reflected back due to different acoustic impedances of tissues and collected to form an image~\cite{Chan2011}. Although particularly useful for imaging structures in motion, ultrasound is not used for brain imaging because of the high attenuation of the sound waves by the skull. \ac{MRI} is also non-ionizing, using high-energy magnets to obtain structural information from changing spin properties of subatomic particles\cite{Plewes2012}. It is one of the most exciting modalities since it can achieve high-resolution scans of the entire body. The drawback is that these \ac{MRI} machines are immense, extremely expensive, and require the user to be immobile during use, which limits its impact to investigations of immobile functions and to populations with the economic resources to access them. We are in need of an imaging technique that is non-invasive, non-ionizing, can be used to diagnose various areas of the body, and is portable and low-cost enough for use by the masses.

% WHY OPTICAL IMAGING. WHY NIR LIGHT?
Optical imaging is a non-invasive and non-ionizing method that uses visible and near-infrared light to probe the molecular function of tissues~\cite{Nunez2018}. Although some optical imaging methods use agents (such as fluorescence and phosphorescence imaging), in this dissertation we focus on non-invasive (no agent) methods of optical imaging. The near-infrared part of the electromagnetic spectrum is typically used because soft tissues show less scattering and absorption to these bands. The composition of the imaged tissue determines how the light is absorbed, reflected, or scattered. The complexity of the interaction between light and tissue was once incredibly difficult to model, but technological advances in computational methods and devices has positioned \ac{NIR} imaging as a contender for functional medical imaging.

% WHAT IS THE BIG CHALLENGE WE ARE ADDRESSING?
Our overarching goal is to demonstrate how \ac{NIR} imaging is a conduit for medical imaging innovations for the rest of the twenty-first century. To do that, we must be bold---we will address modern national and global grand challenges to show the potential breadth of application of \ac{NIR} imaging. The first challenge is posed by the \ac{USAID} through their Saving Lives at Birth Initiative. The goal is to address the heightened high-risk period for babies from the onset of labor through 48 hours after birth in \ac{LMIC}. This period accounts for 48 percent of maternal deaths and 54 percent of neonatal deaths annually~\cite{SLAB2021}. For the second challenge, we turn toward the brain. The \ac{BRAIN} Initiative is focused on the development and application of new technologies to image the brain for the treatment, cure, and prevention of brain disorders. Through funding from the \ac{NIH} \ac{NIBIB}, we will develop a portable neuroimaging system with features tailored towards use in natural environments. And finally, we will address the challenge of improving breast cancer diagnosis and prevention of unnecessary biopsies through a grant from the \ac{NIH} \ac{NCI} for the development of an optical mammography system that augments existing x-ray mammography systems and scans. Although the field of medical imaging is continually advancing, at the time of writing, no contemporary imaging technique is suited to address all three aforementioned challenges. 


% WHAT IS THE SCOPE. WHAT IS INCLUDED/NOT INCLUDED?
This dissertation will show the potential of \ac{NIR} imaging to address a variety of current application-, user-, and setting-specific needs through the development of multiple \ac{NIR} systems. Although each of the imaging systems described in this thesis will vary in attributes (such as complexity, cost, and scalability), as the title of this thesis suggests, we will focus on the following requirements:
\begin{enumerate}
  \item Each \ac{NIR} system must address portability, either through a stand-alone system or through simple integration into an existing imaging modality system. 
  \item Each \ac{NIR} system must be non-invasive (use no reactive agents) and non-ionizing.
  \item Each \ac{NIR} system must utilize the visible and/or near-infrared spectral window.
\end{enumerate}

To address these grand challenges while meeting the requirements above, some system will leverage computational improvements of light propagation models while other systems will integrate technological advancements in sensors to improve existing techniques. In all we will take a product-focused lens to ensure what we are building is addressing the needs of users (and prevent us from falling into the academic pitfall of building for the sake of building). By demonstrating use cases and designs across a variety of medical imaging attributes, we hope to show the medical community at large the benefits of non-invasive \ac{NIR} methodologies and ways to translate these technologies outside of the research setting. 

% WHAT ARE THE AIMS?
%\section{Aims and Objectives}
This thesis is separated into five aims. The first three aims refer to the development of three individual portable and/or wearable near-infrared imaging systems. We will present the design, fabrication, and characterization of these systems as well as measurements on human test subjects. The fourth aim refers to the validation of our systems through characterization with optical phantoms of known optical properties. Finally, the fifth aim condenses the work into a Pugh chart~\cite{Pugh1981} by comparing all three developed \ac{NIR} systems to an elementary \ac{NIR} imaging system, a finger-clip-based pulse oximeter. 

While this introductory chapter sets the challenge and scope of the research for this dissertation, Chapter~\ref{chap:background} gives necessary background into the basics of optical imaging, details the \ac{NIR} imaging techniques used in this work, and defines the ``ilities''~\cite{DeWeck2012} that will be compared between all three systems. Chapter~\ref{chap:moxi} shows how we address the first challenge through the development of a mobile-phone-based pulse oximeter that leverages the sensors inside already ubiquitous mobile phones in \ac{LMIC}. Chapter~\ref{chap:mobi} addresses the second challenge of advancing neuroimaging through the development of a wearable functional brain imaging system with features tailored towards its use in natural, unrestricted environments. The third challenge is addressed in Chapter~\ref{chap:omci}. By combining the physiological measurements from optical imaging with the structural imaging from x-ray, we not only improve stand-alone optical imaging reconstructions but also improve existing x-ray mammography, all without exposing a patient to more ionizing radiation. Chapter~\ref{chap:3dprint} discusses the use of additive manufacturing in the development of optical phantoms utilized by all three systems in the first three aims. Finally, in Chapter~\ref{chap:conclusion}, we compare the three systems across their -ilities and conclude the significance and impact of this work.


% --- EOF ---
