% background.tex:

\chapter{BACKGROUND} % all caps please
\label{chap:background}
This is where the background goes. 



%%% Section
\section{Basics of Optical Imaging}
\label{chap:background:basics}

\subsection{Light-Tissue Interactions}
Biological optical imaging has the capability to detect biological structure, function, and molecular characteristics based on photon interactions with tissue~\cite{Wang2009}. The interaction of light with tissue is governed primarily by three processes: reflection, scattering, and absorption~\cite{Welch2010}.

\begin{figure}
    \begin{center}
    \includegraphics[width=.35\textwidth]{fig/background/lightinteraction.pdf}
    \end{center}
    \caption{Possible interactions when light interfaces with tissue. The pink rectangle represents tissue. White circles are scatterers. Black dots are absorbers.  (1) Light reflects without entering the tissue. (2) Light immediately gets absorbed. (3) Light scatters multiple times before being absorbed. (4) Light scatters multiple times before exiting the tissue on the opposite side it entered. (5) Light scatters multiple times but exits on the side it entered. } 
    \label{fig:lightinteraction}
\end{figure} 

The index of refraction, $n$, is a unitless number that describes how fast light travels through material~\cite{Wang2009}. It is used to determine how much the path of light is bent upon transitioning from one material to the next.  This is governed by Snell’s Law of Refraction~\cite{Wang2009}, $n_1 \times sin\theta_1= n_2\times sin\theta_2$, which define the angle of incidence, $\theta_1$, and angle of refraction, $\theta_2$, based on two media with indices of refraction $n_1$ and $n_2$. Thus, from Snell’s Law, we can also determine the amount of light that is reflected when reaching an interface (Figure~\ref{fig:lightinteraction}). 

Once photons enter a turbid media, they move in all directions and may be scattered or absorbed (Figure~\ref{fig:lightinteraction}). Absorption depends on the component concentrations of tissue~\cite{Nunez2018}. In the visible to near-infrared wavelength range, the primary absorption components include water, hemoglobin, pigment, and lipid~\cite{Du2006, Pogue2006}. The absorption coefficient, $\mu_a$ [$cm^{-1}$], is defined such that, when a photon propagates over an infinitesimal distance $ds$, the probability of absorption is $\mu_a\times ds$~\cite{Welch2010}. The absorption coefficient depends on the molar extinction coefficient of a given chromophore $\epsilon$ [$cm^{-1}\times M^{-1}$], and its Molar concentration, $c$. Thus, the absorption coefficient per wavelength is 

\begin{equation}
    \mu_a(\lambda) = log(10) \sum_{i=1}^{t}\epsilon_i(\lambda)\times c_i~\cite{Nunez2018}. 
\end{equation}
where t is the total number of absorbing components in the tissue. From this, we deduce that $1/\mu_a$ is the average path length traveled by a photon before being absorbed. 

Light entering a tissue can also undergo scattering events, events during which directionality changes occur due to biological structures within the media (Figure~\ref{fig:lightinteraction}). In the visible to infrared wavelength range, the primary scattering components in biological tissue are protein, fat, and mitochondria~\cite{Du2006, Pogue2006}. Analogously, the scattering coefficient, $\mu_s$, is defined such that, when a photon propagates over an infinitesimal distance $ds$, the probability of scattering is $\mu_s\times ds$~\cite{Welch2010}. Additionally, we model the probability distribution of scattered photons by an angular function known as the anisotropy factor, $g$~\cite{Wang2009}. Since $g$ is based on the scattering angle, the closer to 1.0 $g$ is, the more likely the photon is to be scattered in the forward direction. To account for this anisotropy factor, we define the reduced scattering coefficient, $\mu_s^{'}$, as 
$\mu_s^{'} = \mu_s(1-g)$~\cite{Wang2009}. The average distance traveled by a photon between scattering events is $1/\mu_s$.

\subsection{Components of Optical Measurement Systems}
Optical systems are composed of three elementary blocks: a source that radiates light, a sample through which light propagates, and a detector that measures the light intensity after photons have traveled through the sample~\cite{Webster2010}. Although there are numerous types of sources and detectors, here, we highlight only the types used in the optical systems developed for this thesis. 

\ac{LED} are devices that radiate light when a current passes through them~\cite{Webster2010}. They are ubiquitous in modern electronics due to being inexpensive and requiring minimal power to operate. In our \ac{MOXI} system, we leverage the white LEDs used for flash photography common in most smartphones. Our \ac{MOBI} system uses dual-wavelength \ac{LED}s chosen to optimize propagation within the brain layers. Arrays of \ac{LED}s are used in conjunction with digital micromirrors to project color images from projectors. Our \ac{OMCI} system uses an \ac{LED} projector to shine patterns to scan the surface of the breast. For certain applications, it is better to have a light source that does not spread out much, such as \ac{LASER}. Laser light sources produce very narrow beams of light. In \ac{OMCI}, we use a laser to input light into a projector to project patterns onto the breast. 

Detectors are devices used to measure light. Photodiodes are the reverse of \ac{LED}s---they convert light into electrical current~\cite{Webster2010}. Their cost tends to be relative to their sensitivity. \ac{MOXI} and \ac{MOBI} use inexpensive photodiodes chosen to be sensitive to the wavelengths of their associated \ac{LED}s. \ac{OMCI} uses cameras to detect the reflection and transmission of projected patterns. These cameras capture light through a small lens using a tiny array of microscopic detectors. 

The measured light, in combination with the known type of source, allows for the determination of biological structure, function, and molecular characteristics of the tissue through which the light propagated. For example, the detection of photons from particular wavelengths allows us to compute concentrations of oxygenated ($HbO_2$) and de-oxygenated ($HbR$) hemoglobin in tissue. From this, we can infer parameters such as total hemoglobin concentration and tissue oxygen saturation~\cite{Nunez2018}. 
%From this, we can infer parameters such as total hemoglobin concentration (THC=HbR+HbO_2) and tissue oxygen saturation (S〖tO〗_2=  (HbO_2)⁄((HbR+ HbO_2)))[3]. 

\subsection{Optical Phantom Fabrication}
Phantoms are objects with optical properties that mimic human tissues~\cite{Pogue2006}. They are common for evaluating the performance of \ac{NIR} imaging systems~\cite{Pogue2006}. To mimic \ac{NIR} light propagation due to components within biological tissue, phantoms typically attempt to mimic the reduced scattering coefficient ($\mu_s^{'}$) and the wavelength-dependent absorption coefficient ($\mu_a$) in biological tissue~\cite{Dempsey2017}. Traditionally, these phantoms are created using recipes that involve a mix of scattering agents and absorbing pigments with a base~\cite{Hebden1995,Dong2015}. The geometry of the phantom is typically created using either mold casting~\cite{Hahn2012,Mobashsher2014} or spin coating~\cite{Park2013}. While useful for simple phantoms, these methods fall short in supporting complex geometries needed for phantoms requiring structural and physiological properties, such as when DOT is used to image the brain~\cite{Hebden2002,Villringer1997}. Thus, a new method to manufacture phantoms with spatially varying optical properties and anatomically accurate geometries is needed to support the system development, calibration, and testing of new imaging protocols~\cite{Cerussi2012,Diep2015}.  


%%% Section
\section{Imaging Modalities}
\label{chap:background:modalities}
\subsection{Pulse Oximetry}
Pulse oximetry is used to measure oxygen saturation of hemoglobin in arterial blood and is so widely prevalent it is regarded as the fifth vital sign in medical care~\cite{Neff1988}. It is based on two principles. The first is that $HbO_2$ and $HbR$ absorb red and \ac{IR} light differently~\cite{Bohn2015}. Because of this, pulse oximeters tend to emit two wavelengths of light. Traditional (finger-clip) pulse oximeters place light sources and detectors on opposite sides of the finger. The second principle is that arterial blood volume fluctuates with the cardiac cycle while blood volume in veins, capillaries, skin, fat, and bone remains relatively constant~\cite{Sinex1999}. Thus, light that propagates through the finger and is detected by the detector has two components during temporal measurements of the cardiac cycle---a relatively stable and non-pulsatile \ac{DC} component from the constant volume in veins and capillaries, and a pulsatile \ac{AC} component from the volume fluctuation of the arteries~\cite{Lopez2012}. This detected time trace is called a \ac{PPG}~\cite{Sinex1999}. 


\subsection{Functional Near-Infrared Spectroscopy}
\subsection{Diffuse Optical Tomography}
\subsection{Structured-Light Imaging}



%%% Section
\section{``-ilities'' of Near-infrared Imaging Systems}
\label{chap:background:ilities}



%%% Section
\section{Thesis Aims}
\label{chap:background:aims}


% --- EOF ---