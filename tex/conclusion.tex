\chapter{CONCLUSION} % all caps please
\label{chap:conclusion}

\section{MOXI}

\section{MOBI}
We have developed a MATLAB-based modular probe design toolbox, \ac{MOCA}, with the goal of providing \ac{fNIRS} developers with a systematic yet easy-to-use software platform to navigate the large design space of modular \ac{fNIRS} probes and provide metrics-based guidance. \ac{MOCA} simplifies the design problem with module-level parameters such as size, shape, and optode layout as well as probe-level parameters such as the maximum source-to-detector separation and ROI geometry to characterize a modular probe. It offers the ability to perform operator-guided sweeping of probe parameters such as orientation, spacing, and module staggering offset, helping designers explore alternative designs that potentially improve upon existing probes or outline spectra of trade-offs. \ac{MOCA} is quantitative, guided by application-specific \ac{fNIRS} performance metrics, including channel distribution, average brain sensitivity, and spatial multiplexing groups, making it possible for quantitative characterization and comparison between various design decisions. Applying \ac{MOCA} in several case studies, we identified several valuable design considerations that have not been widely recognized, including the importance of fine-tuning module orientation, spacing, and staggering distance. In the meantime, these case studies also demonstrate the complexity of modular probe optimization, where multiple variables compete and eventually lead to alternative designs with various trade-offs. While \ac{MOCA} was not designed to provide full automation for complex probe design and optimization, it offers \ac{fNIRS} probe designers a suite of powerful tools, including module tiling, routing, re-orientation, and fine-tuning of module spacing and staggering offset, with each outcome quantified by meaningful performance metrics. \ac{MOCA} is expected to attract more research interests towards developing next generation modular \ac{fNIRS} systems.

We have designed and validated a lightweight, fiberless, diamond-shaped modular \ac{fNIRS} system with features tailored toward use in natural environments. Its shape and flexible-circuit-based design allow for the system to conform to the scalp and increase optode-to-scalp coupling during use. A dense communication network allows for the automatic determination of the connection topology for the creation of spatial multiplexing groupings to increase the full frame rate of any probe configuration relative to sequential encoding methods. The use of 3-D orientation sensors allows for the determination of optode positions without the need for external hardware. Our \ac{MOBI} modules were validated against a commercial system in cuff occlusion and serial subtraction tests. The \ac{MOBI} \ac{fNIRS} system directly addresses the ergonomic, automatic, and usability concerns necessary for the adoption of wearable systems by the masses. 

\section{OMCI}
In summary, we have developed and validated a low-profile, low-cost, and robust \ac{SLI}-based breast surface acquisition system that can be used in confined low-light mammography-like settings to obtain 3-D breast surfaces. Once calibrated, our \ac{SLI} system can achieve sub-millimeter accuracy with a data acquisition time of less than 40 seconds. We quantified the impact of breast surface estimation methods on \ac{DOT} optical property reconstruction accuracy of inclusions embedded at various depths and found that obtaining accurate breast surfaces is important for \ac{DOT} reconstructions of shallow lesions with a depth less than 25~mm. While contour-extrusion based approaches are relatively simple and produce acceptable reconstructions for deeply embedded tumors, they can result in 30\% to 100\% higher errors when reconstructing shallow tumors. We want to particularly mention that a compact breast shape acquisition system that can fit between mammography compression plates can not only help improve parallel-plate breast \ac{DOT} image reconstructions but can also be incorporated into standard x-ray based \ac{DBT} scanners to help improve 3-D \ac{DBT} image reconstructions. Currently, clinical \ac{DBT} image reconstructions are performed without considering the actual breast shape~\cite{Chong2019} and often result in an inaccurate cylindrical quasi-3D breast shape~\cite{Redbird2008}. Explicitly capturing and considering breast 3-D shapes are expected to lead to improved image quality in \ac{DBT} and other model-based breast imaging modalities.



\section{Pugh Chart}

Finally, we revisit the system lifecycle properties defined in Table~\ref{tab:ilities}. We use the Pugh method~\cite{Pugh1981} to qualitatively rank each of the three \ac{NIR} systems against a reference design using the ilities as the set of criteria. The reference \ac{NIR} system is a standard finger-clip-based, two-wavelength pulse oximeter. Each ility can vary between $\pm3$ indicating that the system is ranked better ($+$), worse ($-$), or the same as ($0$) the reference design. A value of 3 allows for each of the three systems to all be ranked better (or worse) than the reference design while still providing relative ranking between the three systems. The results of this ranking are shown in Table~\ref{tab:pughtable}. 

\begin{table}
\centering
\caption{Pugh Chart ranking of NIR systems}
\label{tab:pughtable}
%\resizebox{\textwidth}{!}{%
\begin{tabular}{@{}lcccc@{}}
\toprule
Ility Name        & Pulse Oximeter & MOXI & MOBI & OMCI \\ \midrule
Adaptability      & 0              & 0    & 3    & 1    \\
Affordability     & 0              & 2    & -1   & -3   \\
Comfortability    & 0              & -1   & 3    & -3   \\
Conformability    & 0              & -1   & 3    & -1   \\
Extensibility     & 0              & 2    & 1    & 3    \\
Interoperability  & 0              & 0    & 3    & 2    \\
Maintainability   & 0              & 3    & -1   & -3   \\
Manufacturability & 0              & 3    & -1   & -3   \\
Modifiability     & 0              & 1    & 2    & 3    \\
Operability       & 0              & -1   & -2   & -3   \\
Portability       & 0              & 3    & -1   & -3   \\
Reconfigurability & 0              & 1    & 2    & 0    \\ \bottomrule
\end{tabular}%
%}
\end{table}

\begin{description}
   \item[Adaptability] \ac{MOXI} is designed for a specific function and is not easily adaptable for other vital signs. \ac{OMCI} can be adapted for other applications. For example, the \ac{SLI} system can be used on its own for surface estimation of other body parts (e.g. facial landmark identification). It is \ac{MOBI}, however, that is ranked highest dues to its modular design that allow for spectroscopy or \ac{DOT} applications at various sites. 
   
   \item[Affordability] \ac{OMCI} is clearly the most expensive \ac{NIR} system we built due to its numerous subsystems and expensive hardware. Although \ac{MOBI} has similar optical components as a traditional pulse, it does use more complex electronics and interfaces that drive up the cost. In contrast, the \ac{MOXI} is more affordable that traditional pulse oximeters since it only requires a small piece of paper. The reason for give it a rank of three is because \ac{MOXI} still requires a smartphone, which a user may or may not have in their possession. 
  
   \item[Comfortability] The silicone covers and wireless capability of the \ac{MOBI} modules allow them to be used for hours at a time. On the other hand, \ac{OMCI} requires heavy compression of the breast to minimize the thickness between paddles. This is so uncomfortable that we have to limit the time in compression to less than 3 minutes. The \ac{MOXI} system, although highly portable, requires the user to actively press onto the camera phone, which can cause discomfort over long-term use compared to the passive design of a traditional finger-clip pulse oximeter. 
   
   \item[Conformability] The reflectance-based design of \ac{MOXI} relies on the flat surface of a phone camera that is susceptible to motion. \ac{OMCI}, like \ac{MOXI}, uses flat surfaces that compress the breast preventing motion. However, the mechanical principle of compressing tissue using two fixed-shaped surfaces is identical to a traditional pulse oximeter in the sense that neither adjusts to different user shapes. In contrast, the flexible-circuit-based \ac{MOBI} modules conform easily to the scalp. 
   
   \item[Extensibility] Given the context, our \ac{OMCI} system be easily extended to include features from state-of-the-art \ac{DOT} research including the use of optimal wide-field illumination patterns and sizes, new \ac{SLI} illumination patterns, and compression-sensor-based tomography.  In contrast, \ac{MOBI} would leverage features from portable \ac{fNIRS} systems, which rely on the use of new driving electronics and optodes, which require new circuit designs. It, however, unlike a pulse oximeter, vary the intensity of light to accomodate hair artifacts. \ac{MOXI} can do software update easily, but features to support other vital signs necesitate specific electronics that the mobile-phone in use may not have. 
   
   \item[Interoperability] \ac{MOXI} is not better or worse in its ability to operate with other imaging systems. By design, \ac{OMCI} is capable of being integrated in existing x-ray mammography systems. However, \ac{MOBI} receives the highest score due to the auxiliary input of the master module, allow its measurements to synchronize with any other system that can output a \ac{TTL} signal. 
   
   \item[Maintainability] Our \ac{MOXI} system is ranked highest because it can be easily maintained with regular software updates and replacing its inexpensive pieces of paper. Our \ac{MOBI} modules are robust and designed to be used in natural settions. However, they were ranked lower than the reference design due to the high number of components (flat-flex cables, caps, master modules) that can potentially break and require replacement. Due to the complexity of our \ac{OMCI} system, it is ranked the lowest in maintainability.
   
   \item[Manufacturability] \ac{MOBI} modules have very similar optical components to a finger-clip pulse oximeter, thus same expertise used in designing the circuit and fabricating the physical enclosure of a pulse oximeter clip is needed for fabricating a \ac{MOBI} module. We ranked them $-1$ because a \ac{MOBI} system also requires the fabrication of the master and trigger boards, as well as the creating of the headgear that holds the modules in place. Our \ac{OMCI} system requires not only fabricating circuits but also mechanical assemblies and sensitive optical fibers. In contrast, \ac{MOXI} simply requires a piece of colored paper. 
   
   \item[Modifiability] Modifiability refers to the ability to change a default set of specified parameters. Besides the color of the paper filter, \ac{MOXI} does not allow any user changes. On our \ac{MOBI} system, a use can change the source currents, detector gains, and sampling strategy (sequentialy and spatial multiplexing). \ac{OMCI} receives the highest ranking in this category due to the ability to change the wide-field and \ac{SLI} patterns, position of the \ac{RF} source location, and scaling factor sensitivity. 
   
   \item[Operability] Although a lot of consideration was taken into the usability of the software \ac{GUI} of our \ac{OMCI} system, it is clear from the extensive training that was required to obtain human subject data that the complexity of all the subsystems and calibration steps prior to acquisition make this system difficult to use, even for a knowledgeable user. Although conceptually similar to a pulse oximeter, our \ac{MOBI} modules require relatively longer setup times to connect modules and affix a cap onto a user. Our \ac{MOXI} system only requires user input into a very basic application. Also easy to operate, it is ranked less than zero because a finger-clip-based pulse oximeter requires no user input.
   
   \item[Portability] Our \ac{MOXI} system receives the highest ranking because it only requires a piece of paper and its Moximeter application can be easily downloaded. Although wearable and portable, compared to a traditional pulse oximeter, our \ac{MOBI} system requires the transportation of multiple modules and supporting electronics. Our \ac{OMCI} breast imaging system is much less portable than a pulse oximeter due to its size and weight.
   
   \item[Reconfigurability] \ac{OMCI} requires a well-aligned and calibrated system to function. In theory, \ac{MOXI} can use different colored paper filters and the Moximeter application attempts to account for the misplacement of the filter on the camera. \ac{MOBI} is by far the most reconfigurable of the three \ac{NIR} systems by simply reconnected the modules in different arrangements. However, the optode layout within a \ac{MOBI} module is fixed, which is why the rank is set to two. 
\end{description}


\section{Future Outlook}