% omci.tex:

\chapter{OPTICAL MAMMOGRAPHY CO-IMAGER (OMCI)} % all caps please
\label{chap:omci}
% should I include the abstract as the first paragraph of the chapter? That way it feels like a stand-alone piece?  

% challenge and application
%innovation and significance

\section{Introduction} %significan
\label{chap:omci:introduction}
Breast cancer is the most commonly diagnosed cancer in women worldwide with an estimated 1,918,030 new cases in 2022 in the United States alone~\cite{Siegel2022}. X-ray mammography is the primary breast cancer screening technique~\cite{Secretan2015} used for early detection to reduce mortality rates~\cite{Tabar2003}. Despite its recommendation for screening, x-ray mammography suffers from a high false-positive diagnostic rate~\cite{Tabar2003, Elmore1998}. The technique lacks both strong structural contrast between healthy and tumor tissue and the ability to quantify tissue functions to assess benign versus malignancy~\cite{Leff2008}. These limitations have led researchers to investigate using diffuse optical tomography (DOT) techniques to characterize the breast tumor's physiology to lower false-positive diagnoses. 

Unlike x-ray mammography, DOT is an imaging modality that uses non-ionizing near-infrared (NIR) radiation to yield three-dimensional (3-D) maps of the optical properties of illuminated tissue~\cite{Boas2001, Dehghani2009, Yamada2014, Hoshi2016}. Biological tissues' primary absorbers in the spectral window from around 600 to 1000~nm have relatively low absorption, allowing NIR light to penetrate through centimeters of tissues~\cite{Gibson2005}. This allows the quantification of physiological properties such as hemoglobin concentration, blood volume, and blood oxygen saturation~\cite{Leff2008, Boas2001}. Malignant tumors generally demand a greater blood supply than their surrounding tissues, leading to increased light absorption that can be delineated using spectroscopy and imaging methods, making DOT particularly useful for breast cancer imaging diagnosis~\cite{Wang2022, Vavadi2014, Flexman2013, Choe2009, Taroni2005}. Additionally, the low spatial resolution of DOT~\cite{Li2010} can be improved by a multi-modal approach with x-ray mammography~\cite{Zimmermann2017, Deng2015, Deng2015a, Fang2009a}. DOT images are known for low spatial resolution largely caused by the high scattering properties of biological tissues~\cite{Boas2001}. The high scattering present in the breast tissue redirects photons to traverse large overlapping probing volumes before their detection, greatly reducing the locality of the measurements and resulting in blurry images. Mathematically, this is reflected as the severe ill-posedness of the inverse problem. Parallel-plate compression of breast tissues has been used in an x-ray mammography scan to minimize overlapping tissues and has also been explored for a number of standalone~\cite{Choe2009, Culver2003} and multi-modal DOT breast imaging systems~\cite{ZhuReview2020, Fang2009a, Krishnaswamy2012}. Obtaining breast surface information to aid quantitative analysis of imaging data has garnered interest from a number of applications, including digital breast tomosynthesis (DBT)~\cite{Rodriguez2017} and magnetic resonance imaging (MRI) scans~\cite{Pallone2014, Ortiz2012}.

For multi-modal DOT systems, the 3-D shape of the breast can be estimated using the structural imaging modality such as DBT~\cite{Fang2011} or MRI~\cite{Brooksby2006}. When a 3-D imaging modality is not available, two-dimensional (2-D) mammography~\cite{Deng2015a} has also been used to provide the shape information. In such case, a simple way to recover a 3-D breast surface is to extrude the 2-D breast contour along the compression axis~\cite{Kruger2013, Kalbhen1999}, or sweep the 2-D breast contour along the contour line extracted from an orthogonal view~\cite{Kita1998}. These methods either rely on assumptions about the 3-D location of certain features (e.g. mamilla position) or assume a constant curvature of the breast along the sweeping direction. For more accurate reconstructions of tissue optical properties, especially near the surface, measuring 3-D breast surface accurately can be greatly beneficial.

Accurately acquiring breast 3-D surface shapes has gained clinical acceptance due in large part to the plastic and reconstructive surgery communities~\cite{Chang2015, Losken2005}. The two prominent techniques for 3-D breast surface imaging are stereophotogrammetry and laser scanning~\cite{Yang2015}. Stereophotogrammetry works by overlaying multiple images of an object taken from different view angles and triangulating the location of the object based on matching features in the multiple images~\cite{Ju2016, Galdino2002, FangOSA2012}. In addition to requiring multiple cameras to increase accuracy~\cite{Henseler2012}, this technique is also heavily influenced by lighting conditions since features need to be extracted from multiple viewpoints~\cite{Henseler2011}. Another limitation is the ``ptosis error'' seen during scanning of ptotic or larger breasts~\cite{Nahabedian2003}. This arises due to the small field of view of stereophotogrammetry systems, leading to inaccuracies in breast surface and volume estimations due to the portions of the breast that are not illuminated. Laser scanning is a surface imaging technique in which rays from a laser beam are reflected off an object and detected by a detector~\cite{Kovacs2006}. Although laser-based acquisition systems can produce more accurate surfaces~\cite{Kovacs2006b}, these systems tend to be expensive~\cite{Kovacs2007, Koch2011} and require the need for very precise setups~\cite{Thomson2009}. Recently, the use of patterned-lasers and orientation-sensitive detectors has led to an increase in portable 3-D laser scanning devices~\cite{Kuzminsky2012}. While low-cost laser-based depth sensors have been widely deployed in game consoles such as Xbox or PlayStation, they are only designed to achieve relatively low spatial accuracy compared to dedicated 3-D scanners. Still, patterned-laser-based surface acquisition systems generally require a minimum scanner-to-target distance of 35~cm~\cite{Ametek2002, Artec3D2022}. Additionally, their typical housing is too large to fit between mammography compression plates~\cite{Artec3D2022, Pallone2014}. Bulky instrumentation and long minimum working distance requirements make stereophotogrammetry and laser scanning techniques infeasible in the confined, low-light mammography setting. 

Another widely used 3-D surface acquisition technique is structured light imaging (SLI)~\cite{Yang2020, Zhang2018}. SLI works by illuminating an object with two-dimensional spatially varying patterns with a projector and capturing images from the illuminated object using cameras~\cite{Geng2011}. The distortion of the specially designed patterns provides information regarding the shape of the object. Calibration of the camera-projector system is easily done by capturing images of a known planar pattern (e.g. a checkerboard). With the ability to use off-the-shelf components, a simple setup with a single projector and camera, and a robust and simple calibration method, SLI is positioned to provide accurate, fast, and cost-effective breast surface scanning~\cite{Yang2020}. However, similar to most patterned-laser surface scanners, commercially available SLI systems have long minimal working distance requirements and large assemblies that cannot readily fit within the confined mammography compression plates~\cite{Zhang2018, Ruiz2017}.

In this work, we have developed a low-profile dual-camera SLI breast shape acquisition system specifically tailed for use in the confined space between parallel breast compression plates. This system can be incorporated with standalone DOT breast scanners or multi-modal DOT systems combined with mammography or DBT, with a minimal scanner-to-target distance between 10 and 15~cm. In the following sections, we first describe the design of the SLI breast scanner and detail the methods for adaptive illumination for subject-specific skin tones as well as approaches to reduce specular reflection from the compression plates. We then compare several traditional surface acquisition methods that leverage mammography images against our SLI-based breast surface acquisition system and quantify the impact of improved breast surface acquisition on the recovery of breast lesions using a series of simulations.



\section{Methods}
\label{chap:omci:methods}



\section{Results}
\label{chap:omci:results}



\section{Discussion}
\label{chap:omci:discussion}



% CONCLUSION GOES INTO THE CONLUSION OF THE ENTIRE THESIS

% --- EOF ---