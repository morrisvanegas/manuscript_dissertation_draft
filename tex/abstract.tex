% abstract.tex:

\begin{abstract}
Medical imaging has become a fundamental research tool that continues to grow at a rapid pace. It has made enormous contributions to accelerating our ability to diagnose malignant breast lesions, measure brain functions, and capture oxygenation vital signs. Unfortunately, contemporary imaging modalities rely heavily on radiation- and nuclear-based methods, require coupling gels or contrast agents, or are simply too massive requiring patients to be immobile during use. These disadvantages not only limit our research to basic functions in highly-controlled environments, but their cost has become a barrier-to-entry for use by many populations that can benefit from their frequent use. 

In order to translate our diagnostic capabilities from the laboratory to clinical and natural settings, we need a non-invasive, non-ionizing, gel-free, portable, and low-cost imaging approach. In this thesis, we focus on optical imaging in the visible and near-infrared (NIR) range. Optical imaging uses multiple wavelengths of light to probe the molecular function of tissues. Its varied implementation methods allow for singular measurements, 2-D functional images, and 3-D tomographic reconstructions. 

Our overarching goal is to demonstrate how NIR imaging is a conduit for medical imaging innovations for the rest of the twenty-first century. We demonstrate this through the design and development of three NIR systems with varying system lifecycle properties to address three national and global grand challenges. From fingertip-based imaging in rural areas, to breast imaging in low-light mammography environments, and even brain imaging in unconstrained locations, these challenges intentionally demonstrate the potential broad application of NIR imaging across populations and settings. Additionally, we facilitate the adoption of NIR imaging through the creation of a software platform for semi-automating the design of new systems based on a modular architecture as well as a systematic method for the creation of optical phantoms using off-the-shelf 3-D printing filaments.  

Overall, this thesis leverages innovations in computational methods, advanced electronic sensors, and ubiquitous devices to elevate existing NIR-based methods into state-of-the-art imaging systems. This work positions NIR-based imaging as a functional medical imaging contender worthy of attention, exploration, and adoption due to its capability to image basic and complex dynamics in a range of environments. 


\end{abstract}

